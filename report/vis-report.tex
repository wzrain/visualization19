\title{Visualizing volcanic eruption events\\Visualization Project Report}
\author{Minchao Li, Zirun Wang, Dexin Yang, Kaifeng Zhao}
\date{\today}

\documentclass[10pt,twocolumn]{article}
\usepackage[margin=0.9in]{geometry}
\usepackage{hyperref}

\begin{document}
\maketitle

\section*{\emph{Abstract}}
\emph{
%Please use this template to write your project report. We are not imposing a strict page limit.
%In addition to the PDF of the report, submit your source code and a video (if appropriate) until 15.12. (22:00).
%%
%Submit the files at:
%\url{https://www.dropbox.com/request/JulBIxu1N0qvkpFWUeBq}
%%
%If the files are larger than 100 MB, provide a download link.
In this project, we focused on visualizing the particles generated by volcanic eruptions in 2011 from Grimsvotn Volcano, Puyehue Volcano and Nabro Volcano to monitor their distribution and moving trend. We got a overview of data by browsing the data interactively. We mainly focused on visualizing pathlines of sulfate aerosols, and their relation between tropopause altitudes and stratosphere. We also did volume rendering about XXX.
}

\section{Introduction}
%Introduce into the topic. Talk about the application. Why is it relevant?
Volcanic eruptions emit harmful particles such as ash and sulfate aerosol into the atmosphere. They not only have an immediate effect of air pollution and aviation shutdown, but also influence the Earth's radiation budget and cause profound climate changes. Therefore, it is significant for people to monitor the states and movements of these particles, which is however really tricky if merely based on simulations. In this project, we integrate visualization methods to analyze what particle clouds look like and how they evolve as time goes after certain volcanic eruptions. We focus on an interesting period of the mid of May to end of July 2011, when three volcanoes (Grimsvoetn Volcano,  Puyehue-Cordon Caulle Compex,  Nabro) erupted, and utilize different types of data measured after these eruptions to conduct visualization and analysis.

\section{Data}
Introduce the data. Who generated it (add references~\cite{Journal,Conference})? Is it simulated or measured? What does the data set contain?
Classify your data: what kind of grids? what kinds of attributes (quantitative, ordinal, nominal)? is it scalar data, vector data?

\section{Goals}
Which goals did you set for your project and which ones did you achieve? Add a time line.

\section{Visualizations}
Show images of your interactive visualization system and describe the story of your visualizations.
Also describe your data preprocessing if applicable.

\subsection{Interactive Data Browsing}
What aspect of the data do you concentrate on? What do you want to show? Justify your choice of the visualization method! How did you implement the visualization? Describe what we see in the visualization. Is the method interactive? What parameters does the method have? What insights did you get from the visualization?

\subsection{Pathlines and Animation}
Same as above.

\subsection{Volume Rendering}
Same as above.

\subsection{Aerosol in Stratosphere}
Same as above.

\section{Contributions}

\subsection{Minchao Li}
What did you contribute to the project?

\subsection{Zirun Wang}
What did you contribute to the project?

\subsection{Dexin Yang}
What did you contribute to the project?

\subsection{Kaifeng Zhao}
What did you contribute to the project?

\section{Discussion}

\subsection{Limitations}
What are the limitations of your approach / your implementation?

\subsection{Future Work}
What would you like to add or study further, if you had more time?


\bibliographystyle{abbrv}
\bibliography{vis-report}

\end{document}
